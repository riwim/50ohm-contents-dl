% Author: Dr. Matthias Jung, DL9MJ
% Year: 2024
% FRAME 1
\begin{circuitikz}
    \begin{scope}[local bounding box=bb]
        \ctikzset{
            diodes/scale=\getDarcImageFactor,
            capacitors/scale=\getDarcImageFactor,
            resistors/scale=\getDarcImageFactor,
            inductors/scale=\getDarcImageFactor
        }
        \draw(0,0) node[american, transformer core](T){};

    
        \draw(T.A1) to [short, -o] ++(0,0) coordinate(ue1);
        \draw(T.A2) to [short, -o] ++(0,0) coordinate(ue2);

        \draw[DARCblue] (ue1) to [open, v^=$U_\text{e}$, draw=DARCblue] (ue2);
        \draw[DARCblue] (T.B1) to [open, v^=$U_\text{s}$, draw=DARCblue] (T.B2);
    
        \draw(T.B1) to [short, -*]++(2,0) coordinate(top);
        \draw(T.B2) to [short, -*]++(2,0) coordinate(bottom);
    
        % Hilfskoordinaten:
        \draw(top)
            to [open, name={hilf}] (bottom);
        \draw(hilf.center)
            to [open] ++(-1,0) coordinate(minus);
        \draw(hilf.center)
            to [open] ++(+1,0) coordinate(plus);
    
        % Links:
        \draw(top)
            to [stroke diode, invert, -*] (minus)
            to [stroke diode] (bottom);
        % Rechts:
        \draw(top)
            to [stroke diode, -*] (plus)
            to [stroke diode, invert] (bottom);
        % Oben:
        \draw(plus)
            to [short] ++(2,0) coordinate(o1)
            to [short,-o] ++(1.0,0) coordinate(o4);
    
        % Unten:
        \draw(minus)
            to [short] ++(0,-2)
            to [short] ++(4,0) coordinate(u1)
            to [short,-o] ++(1.0,0) coordinate(u4);

        \draw[DARCblue] (o4) to [open, v^=$U_\text{a}$, draw=DARCblue] (u4);
    
        % Komponenten:
        \draw(o1) to [R, *-*] (u1);
    
        % Beschriftung:
        \draw(top)    node[anchor=south](){\textasciitilde};
        \draw(bottom) node[anchor=north](){\textasciitilde};
        \draw(plus)   node[anchor=east ](){+};
        \draw(minus)  node[anchor=west ](){--};

        % ROT:
        \draw[DARCred] (T-L2.east)
            |- (T.B1) 
            to [short, i={~}, draw=DARCred] (top)
            to [stroke diode, i={~}, draw=DARCred, fill=DARCred!10, *-*] (plus)
            to [short, i={~}, draw=DARCred] (o1)
            to [R, i={~}, draw=DARCred, *-*] (u1)
            to [short, i={~}, draw=DARCred] (u1-|minus)
            to [short] ++(0,0.5)
            to [short, i={~}, draw=DARCred] (minus)
            to [stroke diode, i={~}, draw=DARCred, fill=DARCred!10, *-*] (bottom)
            to [short] ++(-0.5,0)
            to [short, i={~}, draw=DARCred] (T.B2)
            -| (T-L2.west);

    \end{scope}

    \begin{scope}[shift={($(bb.south west)+(0,-2.25/\getDarcImageFactor)$)}]
        \begin{axis}[
            axis lines=center, 
            axis line style={-Triangle},
            scale only axis,
            axis on top,
            height=2cm,
            width=\linewidth*\getDarcImageFactor,
            xmin=0,
            xmax=2.5*pi,
            ymin=-1,
            ymax=1,
            ticks=none,
            xlabel={$t$},
            xlabel style={anchor=north},
            ylabel={$U_\text{s}$},
            ylabel style={anchor=east},
        ]
        \fill[DARCblue!10] (axis cs:0*pi,-1) rectangle (axis cs:1*pi,1);
        \addplot[smooth, mark=none, DARCblue, thick, domain=0:2.5*pi, samples=400] {0.9*sin(deg(x))}; 
        \end{axis}
    \end{scope}

    \begin{scope}[shift={($(bb.south west)+(0,-4.5/\getDarcImageFactor)$)}]
        \begin{axis}[
            axis lines=center, 
            axis line style={-Triangle},
            scale only axis,
            axis on top,
            height=2cm,
            width=\linewidth*\getDarcImageFactor,
            xmin=0,
            xmax=2.5*pi,
            ymin=-1,
            ymax=1,
            ticks=none,
            xlabel={$t$},
            xlabel style={anchor=north},
            ylabel={$U_a$},
            ylabel style={anchor=east},
        ]
        \fill[DARCblue!10] (axis cs:0*pi,-1) rectangle (axis cs:1*pi,1);
        \addplot[smooth, mark=none, DARCblue, thick, domain=0:2.5*pi, samples=400] {0.9*abs(sin(deg(x)))}; 
        \end{axis}
    \end{scope}

\end{circuitikz}