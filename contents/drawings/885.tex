% Author: Prof. Dr. Matthias Jung, DL9MJ
% Year: 2024
% Colored Version of 193 and 192 combined
\begin{circuitikz}[
    thick,
    longL/.style = {american inductor, inductors/coils=5, inductors/scale=.7, inductors/width=1.25},
    every pin edge/.style={semithick, black}
  ]

  \draw (-3.5,0) -- (3.5,0);
  \fill[top color=gray, bottom color=white] (-3.5,0) rectangle (3.5,-.25);

  \foreach \x in {1.3,1.9,2.5} {
    % right
    \draw[DARCorange]
      (0,5) arc[start angle=180, end angle=0, x radius=\x/2, y radius=.3*\x] -- (\x,0);
    \draw[DARCorange, dash pattern=on 10pt off 5pt, DARCorange]
      (\x,0) arc[start angle=180, end angle=0, x radius=-\x/2, y radius=-.3*\x];
    % left
    \draw[DARCorange]
      (0,5) arc[start angle=0, end angle=180, x radius=\x/2, y radius=.3*\x] -- (-\x,0);
    \draw[DARCorange, dash pattern=on 10pt off 5pt, DARCorange]
      (-\x,0) arc[start angle=0, end angle=180, x radius=-\x/2, y radius=-.3*\x];
  }

  \foreach \x in {1.3,1.9,2.5}
    \draw[DARCgreen] (0,2.5) circle[x radius=\x, y radius=\x/2.5];

  \draw (0,0) node[circ] (gnd) {} to[longL, name={spule}] (0,5) node[circ](end){};

  \node[below=.2 of gnd, fill=white, inner sep=1pt] {Erde};

  \path (-1.9,2.5) arc[start angle=180, end angle=75, x radius=1.9, y radius=1.9/2.5]
    node[coordinate, pin=70:H] {};

  \node[coordinate, pin=0:E] at (2.5,3.5) {};

  \draw(spule.east) -- (end);
\end{circuitikz}