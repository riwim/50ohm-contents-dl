% Author: Stephan Kregel, DG1HXJ
% Year: 2022
\begin{circuitikz}
    \coordinate(init) at (0,0);
    \draw (init)++(3,0) node[npn, name=T1, tr circle] {} ++(0,0);
    \draw (init)
        to [PZ] ++(0,-2) 
        to [vC, tunable end arrow={Bar},invert] ++(0,-2) -- ++(1.5,0) coordinate(c2a) 
        to [C, *-*] ++(0,2) coordinate(c2b)
        to [C, -*] ++(0,2) coordinate(c3b)
        to [R, -*] ++(0,3) coordinate(r1b) -- ++(-1.5,0)
        to [C] ++(0,-1)
        node[rground](){};
    \draw (init)
        -- (c3b)
        -- (T1.B);
    \draw (c2a)
        -- ++(1.5,0) coordinate(r2a)
        to [R, *-*] ++(0,2) coordinate(r2b)
        %-- ++(0,0.5)
         coordinate(c4a)
        to [C, *-] ++(2,0) coordinate(c4b)
        to node[ocirc, name=out]{} ++(0,0);
    \draw (c2b) -- (r2b);
    \draw (r2a)
        to node[rground]{} ++(0,0);
    \draw (r2a)
        -- ++(2,0)
        to node[ocirc, name=gnd]{} ++(0,0);
    \draw (c4a) -- (T1.E);
    \draw (r1b)
        -- ++(1.5,0) coordinate(l1a)
        to [american inductor, *-, name={myL}] ++(2,0) coordinate(l1b)
        node[ocirc, name=vcc]{};
    \draw (T1.C) -- (l1a);
    \draw [dashed] (myL.core west) -- (myL.core east);
    \draw (gnd) node[anchor=north]{--};
    \draw (vcc) node[anchor=south]{+};
\end{circuitikz}