% Author: Dr. Matthias Jung, DL9MJ
% Year: 2022
\begin{circuitikz}[american]
	\ctikzset{
    	resistors/scale=\getDarcImageFactor,
    	capacitors/scale=\getDarcImageFactor,
    	inductors/scale=\getDarcImageFactor,
    	diodes/scale=\getDarcImageFactor,
	}
    \draw(0,0)
        to [european resistor, name={myR}, o-*] ++(2,0) coordinate(o1) 
        to [short] ++(0.5,0) coordinate(o2)
        to [short, -*] ++(0.5,0) coordinate(o3)
        to [short, name={h1}, -*] ++(1,0) coordinate(o4)
        to [short] ++(0.5,0) coordinate(o5)
        to [stroke diode, -*] ++(1.5,0) coordinate(o6)
        to [short,-*] ++(1,0) coordinate(o7)
        to [C,-o] ++(2,0) coordinate(o8);

    \draw(o1)
        to [european resistor, -*] ++(0,-2)
        node[rground](){}
        to [short] ++(0.5,0) coordinate(u2)
        to [short, -*] ++(0.5,0) coordinate(u3)
        to [short, -*] ++(1,0) coordinate(u4)
        to [short] ++(0.5,0) coordinate(u5)
        to [short, -*] ++(1.5,0) coordinate(u6)
        to [short] ++(1,0) coordinate(u7)
        to [european resistor] (o7);

    \draw (myR.north) node[anchor=south]{ZF=\qty{10,7}{\mega\hertz}};
    \draw (o8) node[anchor=south]{NF};

    \draw(o3)
        to [vL, name={myL}, invert, tunable end arrow={Bar}] (u3);
    \draw(o4)
        to [C] (u4);
    \draw(o6)
        to [C] (u6);

    \draw[dashed](myL.core west) -- (myL.core east);

    \draw[dashed, gray] (o2) 
        -- ++(0,0.5) coordinate(h2)
        to [short] (h2-|o5)
        -- (u5)
        -- ++(0,-0.5) coordinate(h3)
        to [short, name={h4}] (h3-|o2)
        -- (o2);

    \draw[gray] (h4.center) node[anchor=north]{$f_\mathrm{res}\neq \mathrm{ZF}$};
\end{circuitikz}