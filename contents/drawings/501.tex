% Author: Dr. Matthias Jung, DL9MJ
% Year: 2022
\begin{circuitikz}
    \ctikzset{
        blocks/scale=\getDarcImageFactor,
        RF/scale=\getDarcImageFactor,
        misc/scale=\getDarcImageFactor
    }
    \draw (0,0)
        node [twoportshape, t={1}](n1){};
    \draw(n1.east)
        node[inputarrow, rotate=180]{}
        to[short] ++(1,0)
        node[inputarrow]{}
        node[cloud, draw, anchor=west, scale={3*\getDarcImageFactor}] (cld) {};
    \draw(cld.north)
        node[anchor=south](){Netzwerk};
    \draw(cld.east)
        node[inputarrow, rotate=180]{}
        to[short] ++(1,0)
        node[inputarrow]{}
        node [twoportshape, t={2}, anchor=west](n2){};
    \draw(n2.east)
        node[inputarrow, rotate=180]{}
        to[short, name={h1}] ++(2,0)
        node[inputarrow]{}
        node [twoportshape, t={3}, anchor=west](n3){};
    \draw(n3.east)
        to[short] ++(1,0)
        node [antenna](ant){};
    \draw(h1.north) ++(0,0.75) node[anchor=south](h2){\footnotesize{Kontrollsignale}};
    \draw(h2.north)            node[anchor=south](h3){\footnotesize{Steuer- und}};
    \draw(h3.north)            node[anchor=south](h4){\footnotesize{Audio (NF) TX}};
    \draw(h4.north)            node[anchor=south](h5){\footnotesize{Audio (NF) RX}};
\end{circuitikz}