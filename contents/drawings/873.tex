% Author: Prof. Dr. Matthias Jung, DL9MJ
% Year: 2024
\begin{tikzpicture}
    \draw[ultra thick, DARCblue, fill=DARCblue!20] (0,0) circle (2);
    \draw[ultra thick, DARCred]  (0,0) circle (3);
    %
    \foreach \angle in {0,20,...,360} {
        \path (\angle:2) coordinate(in\angle);
        \path (\angle:3) coordinate(out\angle);
    }
    %
    \draw[DARCgreen]
        (in80)   --
        (out60)  --
        (in40)   -- 
        (out20)  --
        (in0)    --
        (out340) --
        (in320);
    %
    \draw(in80)  node[antenna, scale=0.25, rotate=-10](){};
    \draw(in80)  node[anchor=north, rotate=-10](){TX};
    \draw(in320) node[antenna, scale=0.25, rotate=-130](){};
    \draw(in320) node[anchor=south, rotate=50](){RX};
    %
    \draw[DARCorange]
        (in80)   --
        (out100) --
        (in120)  --
        (out140) --
        (in160)  --
        (out180) --
        (in200)  --
        (out220) --
        (in240)  --
        (out260) --
        (in280)  --
        (out300) --
        (in320);
        \draw[decorate, decoration={text along path, text align=center, text={Kurzer Weg}, text color=DARCgreen}]  (80:1.70) arc (80:-40:1.70) -- (-40:1.70);
        \draw[decorate, decoration={text along path, text align=center, text={Langer Weg}, text color=DARCorange}] (80:1.75) arc (80:320:1.75) -- (320:1.75);
\end{tikzpicture}%