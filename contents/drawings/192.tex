% Author: Malte, DE7LMS
% Year: 2021
\begin{circuitikz}[
    thick,
    longL/.style = {american inductor, inductors/coils=5, inductors/scale=.7, inductors/width=1.25},
    every pin edge/.style={semithick, black}
  ]
  % original for comparison
  % \node[anchor=south] at (0,-1.35) {\includegraphics[width=7cm]{tb_407_q_1.png}};

  \draw (-3.5,0) -- (3.5,0);
  \fill[top color=gray, bottom color=white] (-3.5,0) rectangle (3.5,-.25);

  \draw (0,0) node[circ] (gnd) {} to[longL] (0,5) node[circ] {};
  \foreach \x in {1.3,1.9,2.5} {
    % right
    \draw
      (0,5) arc[start angle=180, end angle=0, x radius=\x/2, y radius=.3*\x] -- (\x,0);
    \draw[dash pattern=on 10pt off 5pt]
      (\x,0) arc[start angle=180, end angle=0, x radius=-\x/2, y radius=-.3*\x];
    % left
    \draw
      (0,5) arc[start angle=0, end angle=180, x radius=\x/2, y radius=.3*\x] -- (-\x,0);
    \draw[dash pattern=on 10pt off 5pt]
      (-\x,0) arc[start angle=0, end angle=180, x radius=-\x/2, y radius=-.3*\x];
  }

  % Using Bézier curves looks more pleasing but is less correct
  % \foreach \x/\y in {-2.5/1,-1.9/.7,-1.3/.5,1.3/.5,1.9/.7,2.5/1}
  %   \draw (0,5) .. controls +(0,\y) and +(0,\y) .. (\x,5)
  %     -- (\x,0)
  %     .. controls +(0,-\y) and +(0,-\y) .. (0,0);

  % X marks the spot
  \path (-1.9,2.5) arc[start angle=180, end angle=75, x radius=1.9, y radius=1.9/2.5]
    node[coordinate, pin=70:X] {};

  \foreach \x in {1.3,1.9,2.5}
    \draw[dash pattern=on 10pt off 5pt] (0,2.5) circle[x radius=\x, y radius=\x/2.5];

  \node[below=.2 of gnd, fill=white, inner sep=1pt] {Erde};
\end{circuitikz}