% Author: Prof. Dr. Matthias Jung, DL9MJ
% Year: 2025
\begin{circuitikz}
    \tikzset{german switch/.style={
        cute open switch, 
        bipoles/cuteswitch/thickness=0.2,
        bipoles/cuteswitch/shape=circ,
        bipoles/cuteswitch/width=0.5,
    }}
    \ctikzset{diodes/scale=0.6}
    \draw(0.0,6.0)
        node[anchor=south] () {+}
        to [stroke diode, o-, name={a1}] ++(0,-2)
        node[genericshape, anchor=north](resistor){};
    \draw(resistor.south)
        to [short] ++(0,-1) coordinate (a2)
        to [short] ++(0,-0.5)
        node[npn, tr circle, anchor=C] (tr) {};
    \draw(resistor.south west) -- (resistor.north east);

    \draw(tr.E) to [short] ++(0,-0.5)
                node [rground] () {};

    \draw[dashed](tr.B) -- ++(-0.5,0); 

    \draw(a1)
    to [open] ++(1.5,0) coordinate(h1)
    to [german switch, name={switch}] (h1|-a2) coordinate(h2);

    \draw[dashed](h1) -- ++ (0,+1.0);
    \draw[dashed](h2) -- ++ (0,-1.0);
    \draw[dashed](resistor.east) -- (switch.mid);
\end{circuitikz}