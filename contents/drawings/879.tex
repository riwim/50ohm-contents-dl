% Prof. Dr. Matthias Jung, DL9MJ
% 2024
% Alternativbild für Frage ED402. Noch in Abklärung!
\begin{circuitikz}[european]
    \draw(0,0) to [short,o-*] ++(2,0) coordinate(c1)
               to [short, -*] ++(2,0) coordinate(c2)
               to [short,   ] ++(2,0) coordinate(c3)
               to [short, -o] ++(2,0) coordinate(c4);

    \draw(4,3) node[npn, tr circle](trans){};
    \draw(0,3) coordinate(a1) 
               to [C,l={\qty{1}{\micro\farad}},o-*] ++(2,0) coordinate(d1)
               to [short] (trans.B);
    
    \draw(d1) to [R,l={\qty{820}{\ohm}}] (c1);

    \draw(trans.E) to [short] (c2);

    \draw(c2) node[rground](){};

    \draw(d1) to [R, l_={\qty{4,7}{\kilo\ohm}},-*] ++(0,3) coordinate(e1)
              to [short] ++(-1,0)
              to [eC,l_={\qty{10}{\micro\farad}}] ++(0,-1.0)
              node[rground](){}; 

    \draw(e1) to [short, -*] ++(2,0) coordinate(e2)
              to [R, l={\qty{22}{\ohm}}] (trans.C);

    \draw(trans.C) to [C, *-, l_={\qty{100}{\micro\farad}}] ++(3.5,0) coordinate(x1) to [loudspeaker, -*] (x1|-c4);

    \draw(e2) to [short,-o] ++(4,0) coordinate(a2);
    
    \draw(a1) node[anchor=south](){E};
    \draw(c4) node[anchor=west](){--};
    \draw(a2) node[anchor=west](){+};
\end{circuitikz}
