%
% Author: Michael Groni, DB7YI
% Year: 2023
%

% Metall elektrisch neutral

\begin{tikzpicture}

	\tikzstyle{ladung}=[circle, draw, text centered]
	\tikzstyle{plus}=[ladung, fill=DARCred]
	\tikzstyle{minus}=[ladung, fill=DARCblue]

	% Plus oben links
	\node (obenLinks)  at (0,0) [plus] {$+$};
	
	
	% Koordinaten der Ankerpunkte bestimmen
	\pgfpointanchor{obenLinks}{west} % Westlicher Ankerpunkt
	\pgfgetlastxy{\westx}{\westy} % Koordinaten des westlichen Ankerpunkts speichern
  	\pgfpointanchor{obenLinks}{east} % Östlicher Ankerpunkt
  	\pgfgetlastxy{\eastx}{\easty} % Koordinaten des östlichen Ankerpunkts speichern
  	% Abstand zwischen den Ankerpunkten berechnen
  	\pgfmathsetlengthmacro{\nodebreite}{\eastx-\westx}
	
	% Metallgitter
	\foreach \x in {2, 4, 6, 8, 10}
    {
    	\node at (\x*\nodebreite, 0) [plus] {$+$};
    }
    \foreach \x in {0, 2, 4, 6, 8, 10}
    {
    	\node at (\x*\nodebreite, 2.5*\nodebreite) [plus] {$+$};
    } 
	
	
	% $-$ als Beschriftung ergibt einen kleineren Node als $+$.
	% Workaround: minimum width bei negativen Ladungen
	% text height=0.25*\nodebreite zentriert das Minus-Zeichen vertikal
	\node at (-0.2*\nodebreite, 1.5*\nodebreite) [minus, minimum width=\nodebreite, text height=0.25*\nodebreite] {$-$};
	\node at (0.9*\nodebreite, 0.7*\nodebreite) [minus, minimum width=\nodebreite, text height=0.25*\nodebreite] {$-$};
	\node at (1.5*\nodebreite, 1.6*\nodebreite) [minus, minimum width=\nodebreite, text height=0.25*\nodebreite] {$-$};
	\node at (3.05*\nodebreite, 2.9*\nodebreite) [minus, minimum width=\nodebreite, text height=0.25*\nodebreite] {$-$};
	\node at (3.0*\nodebreite, -0.2*\nodebreite) [minus, minimum width=\nodebreite, text height=0.25*\nodebreite] {$-$};
	\node at (2.9*\nodebreite, 1.4*\nodebreite) [minus, minimum width=\nodebreite, text height=0.25*\nodebreite] {$-$};
	\node at (4.7*\nodebreite, 1.0*\nodebreite) [minus, minimum width=\nodebreite, text height=0.25*\nodebreite] {$-$};
	\node at (6.0*\nodebreite, 1.1*\nodebreite) [minus, minimum width=\nodebreite, text height=0.25*\nodebreite] {$-$};
	\node at (7.2*\nodebreite, 1.3*\nodebreite) [minus, minimum width=\nodebreite, text height=0.25*\nodebreite] {$-$};
	\node at (9*\nodebreite, 2.9*\nodebreite) [minus, minimum width=\nodebreite, text height=0.25*\nodebreite] {$-$};
	\node at (9*\nodebreite, 0.9*\nodebreite) [minus, minimum width=\nodebreite, text height=0.25*\nodebreite] {$-$};
	\node at (10.1*\nodebreite, 1.3*\nodebreite) [minus, minimum width=\nodebreite, text height=0.25*\nodebreite] {$-$};
	
	% Rechteck außen herum
	\draw (-\nodebreite, -\nodebreite) rectangle (11*\nodebreite, 3.5*\nodebreite);
\end{tikzpicture}