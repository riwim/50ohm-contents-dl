% Author: Prof. Dr. Matthias Jung, DL9MJ
% Year: 2024
\begin{circuitikz}
    \ctikzset{
        diodes/scale=\getDarcImageFactor,
        capacitors/scale=\getDarcImageFactor,
        resistors/scale=\getDarcImageFactor,
        inductors/scale=\getDarcImageFactor
    }
    \draw(0,0) node[american, transformer core](T){};


    \draw(T.A1) to [short, -o] ++(0,0) coordinate(ue1);
    \draw(T.A2) to [short, -o] ++(0,0) coordinate(ue2);

    \draw[DARCblue] (ue1) to [open, v^=$U_\text{e}$, draw=DARCblue] (ue2);
    \draw[DARCblue] (T.B1) to [open, v^=$U_\text{s}$, draw=DARCblue] (T.B2);

    \draw(T.B1) to [short, -*]++(2,0) coordinate(top);
    \draw(T.B2) to [short, -*]++(2,0) coordinate(bottom);

    % Hilfskoordinaten:
    \draw(top)
        to [open, name={hilf}] (bottom);
    \draw(hilf.center)
        to [open] ++(-1,0) coordinate(minus);
    \draw(hilf.center)
        to [open] ++(+1,0) coordinate(plus);

    % Links:
    \draw(top)
        to [stroke diode, invert, -*] (minus)
        to [stroke diode] (bottom);
    % Rechts:
    \draw(top)
        to [stroke diode, -*] (plus)
        to [stroke diode, invert] (bottom);
    % Oben:
    \draw(plus)
        to [short] ++(2,0) coordinate(o1)
        to [short,-o] ++(1.0,0) coordinate(o4);

    % Unten:
    \draw(minus)
        to [short] ++(0,-2)
        to [short] ++(4,0) coordinate(u1)
        to [short,-o] ++(1.0,0) coordinate(u4);

    \draw[DARCblue] (o4) to [open, v^=$U_\text{a}$, draw=DARCblue] (u4);

    % Komponenten:
    \draw(o1) to [R, *-*] (u1);

    % Beschriftung:
    \draw(top)    node[anchor=south](){\textasciitilde};
    \draw(bottom) node[anchor=north](){\textasciitilde};
    \draw(plus)   node[anchor=east ](){+};
    \draw(minus)  node[anchor=west ](){--};
\end{circuitikz}