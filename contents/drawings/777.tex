% Dr. Matthias Jung (DL9MJ)
% 2022
\begin{circuitikz}
    \coordinate(init) at (0,0);
    \draw (init)++(7,3) node[npn, name=T1, tr circle] {} ++(0,0);
    \draw (init) 
        to [american inductor, inductors/coils=4 , inductors/width=1] ++(0,2) -- ++(0,1) -- ++(1.5,0) to node[circ, name=c1a]{} ++(0,0)
        to [vC,tunable end arrow={Triangle}, invert, o-*]++(0,-4);
    \draw (init) 
        -- ++(0,-1) 
        -- ++(1.5,0) coordinate(c1b);
    \draw (c1a)
        to node[short, .-.]{} ++(1.5,0) coordinate(c3a) 
        to [C, *-*] ++(0,-2) coordinate(c3b)
        to [C, *-*] ++(0,-2) coordinate(c4b);
    \draw (c1b) -- (c4b);
    \draw (c3a)
        to [C, *-*] ++(2,0) coordinate(c2b);
    \draw (c4b) 
        -- ++(2,0) coordinate(r3a)
        to [R, *-] ++(0,2)
        -- (c2b)
        to [R, *-] ++(0,3) node[ocirc, name=r4b]{} -- ++(-2,0)
        to [C] ++(0,-1) node[rground]{};
    \draw (r4b) 
        -- ++(2,0) coordinate(t1a) 
        -- (T1.C);
    \draw (t1a)
        to [R, *-] ++(2,0) 
        node[ocirc, name=vcc]{};
    \draw (c2b)
        -- ++(0.75,0)
        node[ocirc, name=c]{}
        to (T1.B);
    \draw (c3b) -- ++(4,0)
        node[ocirc, name=r2a]{}
        to [R, -*] ++(0,-2) coordinate(r2b)
        node[rground]{};
    \draw (r2a) -- (T1.E);
    \draw (r3a) -- (r2b);
    \draw (r2b) 
        -- ++(2,0)
        node[ocirc, name=gnd]{};
    \draw (gnd) node[anchor=south]{--};
    \draw (vcc) node[anchor=north]{+};
    \draw (c1a) node[anchor=south]{\textbf{A}};
    \draw (r4b) node[anchor=south]{\textbf{B}};
    \draw (c)   node[anchor=south]{\textbf{C}};
    \draw (r2a) node[anchor=west ]{\textbf{D}};
\end{circuitikz}