%
% Author: Michael Groni, DB7YI
% Year: 2023
%

% Anziehung und Abstoßung elektrischer Ladungen

\begin{tikzpicture}

	\tikzstyle{ladung}=[circle, draw, text centered]
	\tikzstyle{plus}=[ladung, fill=DARCred]
	\tikzstyle{minus}=[ladung, fill=DARCblue]

	% Plus - Minus
	\node (obenLinks)  at (0,0) [plus] {$+$};
	
	% Koordinaten der Ankerpunkte bestimmen
	\pgfpointanchor{obenLinks}{west} % Westlicher Ankerpunkt
	\pgfgetlastxy{\westx}{\westy} % Koordinaten des westlichen Ankerpunkts speichern
  	\pgfpointanchor{obenLinks}{east} % Östlicher Ankerpunkt
  	\pgfgetlastxy{\eastx}{\easty} % Koordinaten des östlichen Ankerpunkts speichern
  	% Abstand zwischen den Ankerpunkten berechnen
  	\pgfmathsetlengthmacro{\nodebreite}{\eastx-\westx}
	\draw[-Triangle] (obenLinks.east) -- +(0:1.3*\nodebreite);
	% $-$ als Beschriftung ergibt einen kleineren Node als $+$.
	% Workaround: minimum width bei negativen Ladungen
	% text height=0.25*\nodebreite zentriert das Minus-Zeichen vertikal
	\node (obenRechts) at (4*\nodebreite, 0) [minus, minimum width=\nodebreite, text height=0.25*\nodebreite] {$-$};
	\draw[-Triangle] (obenRechts.west) -- +(180:1.3*\nodebreite);
	
	% Plus - Plus
	\node (mitteLinks)  at (0, -2*\nodebreite) [plus] {$+$};
	\draw[-Triangle] (mitteLinks.west) -- +(180:1.3*\nodebreite);
	\node (mitteRechts) at (4*\nodebreite, -2*\nodebreite) [plus] {$+$};
	\draw[-Triangle] (mitteRechts.0) -- +(0:1.3*\nodebreite);
	
	% Minus - Minus
	\node (untenLinks)  at (0, -4*\nodebreite) [minus, minimum width=\nodebreite, text height=0.25*\nodebreite] {$-$};
	\draw[-Triangle] (untenLinks.west) -- +(180:1.3*\nodebreite);
	\node (untenRechts) at (4*\nodebreite, -4*\nodebreite) [minus, minimum width=\nodebreite, text height=0.25*\nodebreite] {$-$};
	\draw[-Triangle] (untenRechts.east) -- +(0:1.3*\nodebreite);
	
	% horizontale Linie zwischen 1. und 2. Zeile
	\path (obenLinks) -- (mitteLinks) node[midway] (a) {};
	\node [left=2*\nodebreite of a] (b) {};
	\path (obenRechts) -- (mitteRechts) node[midway] (c) {};
	\node [right=2*\nodebreite of c] (d) {};
	\draw[draw=DARCgray] (b) -- (d);
	
	% horizontale Linie zwischen 3. und 4. Zeile
	\path (mitteLinks) -- (untenLinks) node[midway] (e) {};
	\node [left=2*\nodebreite of e] (f) {};
	\path (mitteRechts) -- (untenRechts) node[midway] (g) {};
	\node [right=2*\nodebreite of g] (h) {};
	\draw[draw=DARCgray] (f) -- (h);
\end{tikzpicture}