% Author: Malte, DE7LMS
% Year: 2021

\def\domain{6}
\def\saxis#1{#1/\domain*2, 0, 0}
\def\hfield#1{#1/\domain*2, {1.5 * -sin(#1 r)}, {1.5 * cos(#1 r)}}
\def\efield#1{#1/\domain*2, {1.5 * sin((#1 + pi/2) r)}, {1.5 * -cos((#1 + pi/2) r)}}
\def\saxisPi#1{(\saxis{#1 * pi})}
\def\hfieldPi#1{(\hfield{#1 * pi})}
\def\efieldPi#1{(\efield{#1 * pi})}

\begin{tikzpicture}[
    thick,
    >={Triangle},
    samples=500,
    x={(1cm,0.2cm)},
    y={(0cm,1cm)},
    z={(0.6cm,-0.2cm)},
    domain=0:\domain*pi
    ]

    \fill[lightgray] (0, -2.1, -2) -- (2.2 * pi, -2.1, -2) -- (2.2 * pi, -2.1, 2) -- (0, -2.1, 2) -- cycle;
    \node[left] at (2, -2.1, 0) {Erde};
    \draw[draw=gray, ->] (0, 0, 0) -- (2.35 * pi, 0, 0) node[right] {S};
    \draw[DARCgreen] plot (\hfield\x);
    \draw[DARCorange] plot (\efield\x);
    \path \hfieldPi{0.5} node[below] {H};
    \path \efieldPi{0.5} node[right] {E};
    \foreach \x in {0,0.5,1,...,\domain} {
            \draw[->, DARCgreen!50] \saxisPi\x -- \hfieldPi\x;
            \draw[->, DARCorange!50] \saxisPi\x -- \efieldPi\x;
        }
\end{tikzpicture}