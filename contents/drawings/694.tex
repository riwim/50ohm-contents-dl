% Author: Prof. Dr. Matthias Jung, DL9MJ
% Year: 2024
\begin{circuitikz}
    \tikzset{german switch/.style={
        cute open switch, 
        bipoles/cuteswitch/thickness=0.2,
        bipoles/cuteswitch/shape=circ,
        bipoles/cuteswitch/width=0.5,
    }}
    \ctikzset{
        resistors/scale=0.75,
        inductors/scale=0.75,
        RF/scale=0.75,
    }
    \draw(0,0)
        to [battery2, name={battery}] ++(1,0)
        to [open] ++(1,0)
        to [european resistor, name={resistor}] ++(1,0)
        to [open] ++(1,0)
        to [C, name={capacitor}] ++(1,0)
        to [open] ++(1,0)
        to [american inductor, name={inductor}] ++(1,0);
    \draw(0,-2)
        to [stroke diode, name={diode}] ++(1,0)
        to [open] ++(1,0)
        to [stroke led, name={led}] ++(1,0)
        to [open] ++(1,0)
        to [german switch, name={switch}] ++(1,0)
        to [open] ++(1,0)
        to [open, name={transistor1}] ++(1,0);
    \draw(transistor1.center) node[npn, tr circle](transistor2){};
    \draw(0,-4)
        to [open, name={antenna1}] ++(1,0)
        to [open] ++(1,0) coordinate(ground1)
        to [open, name={ground}] ++(1,0)  coordinate(ground2)
        to [open] ++(1,0)
        to [rmeter, t=V, name={vm}] ++(1,0)
        to [open] ++(1,0)
        to[rmeter, t=A, name={am}] ++(1,0);
    \draw(antenna1.center) node[bareantenna, anchor=center](antenna2){};
    \draw(antenna2.south) --  ++(0,-0.5);
    \draw(ground1) node[rground, anchor=west](){};
    \draw(ground2) node[ground, anchor=east](){};
    %
    \draw(0,-1-|battery.center) node[](){Batterie};
    \draw(0,-1-|resistor.center) node[](){Widerstand};
    \draw(0,-1-|capacitor.center) node[](){Kondensator};
    \draw(0,-1-|inductor.center) node[](){Spule};
    %
    \draw(0,-3-|diode.center) node[](){Diode};
    \draw(0,-3-|led.center) node[](){LED};
    \draw(0,-3-|switch.center) node[](){Schalter};
    \draw(0,-3-|transistor1.center) node[](){Transistor};
    %
    \draw(0,-5-|antenna1.center) node[](){Antenne};
    \draw(0,-5-|ground.center) node[align=center](){Masse und\\Erde};
    \draw(0,-5-|vm.center) node[align=center](){Spannungs-\\messgerät};
    \draw(0,-5-|am.center) node[align=center](){Strom-\\messgerät};
\end{circuitikz} 