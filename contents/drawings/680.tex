% Author: Prof. Dr. Matthias Jung
% Year: 2023
\begin{tikzpicture}
    %\useasboundingbox (-3.75,-1.0) rectangle (10.5,3);
    \node[draw, powershape] at (0,0) (netzteil) {};
    \draw(netzteil.north) ++(0,0.6) node[anchor=south]{Netzgerät};
    \draw(netzteil.south) node[anchor=north]{\textdirectcurrent\qty{13,8}{V}};
    \node[draw, trxshape] at (8,0) (trx) {};
    \draw(trx.east) -- ++(1,0) node[antenna]{};
    \draw(trx.north) ++(0,0.6) node[anchor=south]{Transceiver (TRX)};
    \draw(trx.south) node[ground]{};
    %
    \draw[red](netzteil.plus)
        to [fuse, l={Sicherung}, color=red] ++(2.5,0)
        to [short, i={Gleichstrom (I)}, color=red] (trx.plus);
    \draw(trx.minus) to [short, i={}, l={\textcolor{gray}{\small{\textit{Kurzschluss- und Verpolungsgefahr}}}}] (netzteil.minus);
    \draw(netzteil.u1) to [short, -o] ++(-1,0) coordinate(c1);
    \draw(netzteil.u2) to [short, -o] ++(-1,0) coordinate(c2);
    \draw(c1) to [open, l_={$\sim\qty{230}{\volt}$}] (c2);
    \draw[gray](c1) to [open] ++(-0.5,0.4) node[](c3){\itshape\small Hohe Spannung};
\end{tikzpicture}