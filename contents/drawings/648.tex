% Author: Prof. Dr. Matthias Jung
% Year: 2023
\begin{tikzpicture}
    \tikzset{
        buffer/.style={
            draw,
            very thick,
            shape border rotate=0,
            regular polygon,
            regular polygon sides=3,
            fill=DARCred,
            minimum height=4em,
            rounded corners = 4pt
        }
    }
    \draw[thick](-2,0) -- (22,0);
    \draw[thick] (5,0) -- ++(4,7) -- ++(1,-2) -- ++(2,3) coordinate(top) -- (17,0);
    \draw(top) node[dinantenna, scale=0.75](ant){};

    \draw(0,0) coordinate(b1);
    \draw(b1) ++(1,0) coordinate(b2);
    \draw(b1) to [short] ++(0.5,1) coordinate(geschlechtsteil);
    \draw(b2) -- (geschlechtsteil);
    \draw(geschlechtsteil) to [short] ++(0,1) coordinate(hals);
    \draw(hals) to [short] ++(0.5,-1) coordinate(h1);
    \draw(hals) -- ++(-0.5,-1) coordinate(h2);
    \draw(hals) to [open] ++(0,0.5) circle (0.5);
    \draw(h1) node[draw, rotate=-45, inner sep=0pt, minimum width=10*\getDarcImageFactor, minimum height=20*\getDarcImageFactor, anchor=south](hf1){};
    \draw(hf1.north) node[dinantenna, scale=0.4, rotate=-45, anchor=south](ant1){};
    \draw[DARCred](ant1.top) node [waves, rotate=45, anchor=left](){};
    %\draw[DARCred, thick](ant1.top) -- (ant.top);

    \draw(20,0) coordinate(b1);
    \draw(b1) ++(1,0) coordinate(b2);
    \draw(b1) to [short] ++(0.5,1) coordinate(geschlechtsteil);
    \draw(b2) -- (geschlechtsteil);
    \draw(geschlechtsteil) to [short] ++(0,1) coordinate(hals);
    \draw(hals) to [short] ++( 0.5,-1) coordinate(h1);
    \draw(hals) -- ++(-0.5,-1) coordinate(h2);
    \draw(hals) to [open] ++(0,0.5) circle (0.5);
    \draw(h2) node[draw, rotate=45, inner sep=0pt, minimum width=10*\getDarcImageFactor, minimum height=20*\getDarcImageFactor, anchor=south](hf1){};
    \draw(hf1.north) node[dinantenna, scale=0.4, rotate=45, anchor=south](ant1){};
    \draw[DARCred](ant1.top) ++(-0.25,0.25) node [waves, rotate=-45, anchor=right](){};
    %\draw[DARCred, thick](ant1.top) -- (ant.top);

\end{tikzpicture}