% Author: Malte, DE7LMS
% Modifier: Dr. Matthias Jung, DL9MJ
% Year: 2022
% TH138 λ/4 ground plane
\begin{circuitikz}
    % Corpus:
    \draw (.2,1.4) coordinate(feed right)
        arc[start angle=0, end angle=180, x radius=.2, y radius=.1]
        coordinate(feed left)
        arc[start angle=180, end angle=360, x radius=.2, y radius=.1];
    \draw(feed right) -- ++(0,-2) coordinate(c1);
    \draw(feed left)  -- ++(0,-2) coordinate(c2);
    \draw (c1)
        arc[start angle=0, end angle=-90, x radius=.2, y radius=.1] coordinate (bottom)
        arc[start angle=-90, end angle=-180, x radius=.2, y radius=.1];
    \draw (bottom) -- ++(0,-.2);
    \draw[dashed] (bottom) -- ++(0,2);
  
    % Radials:
    \path (.2,1.4)
        arc[start angle=0,   end angle=45,  x radius=.2, y radius=.1] coordinate (up right)
        arc[start angle=45,  end angle=135, x radius=.2, y radius=.1] coordinate (up left)
        arc[start angle=135, end angle=225, x radius=.2, y radius=.1] coordinate (down left)
        arc[start angle=225, end angle=315, x radius=.2, y radius=.1] coordinate (down right);
  
    \path (down right) ++(2,-1) coordinate (down right end);
    \path[very thick, shorten <=-1pt, {Circle[length=2pt]}-, line cap=round]
      (up right) edge +(2,1)
      (up left) edge +(-2,1)
      (down right) edge (down right end)
      (down left) edge +(-2,-1);
  
    % Antenne:
    \draw (feed left) to [open, name={m1}] (feed right);
    \draw[very thick] (m1.center) -- ++(0,3);
\end{circuitikz}