% Author: Dr. Matthias Jung, DL9MJ
% Year: 2021
\begin{circuitikz}[american]
    \ctikzset{
        resistors/scale=\getDarcImageFactor,
        capacitors/scale=\getDarcImageFactor,
        inductors/scale=\getDarcImageFactor,
        diodes/scale=\getDarcImageFactor,
        misc/scale=\getDarcImageFactor
    }
    \draw(0,0) coordinate(m0)
        to [C, o-*] ++(2,0) coordinate(west)
        to [stroke diode, invert, -*] ++(1.5,1.5) coordinate(north)
        to [stroke diode, invert, -*] ++(+1.5,-1.5) coordinate(east)
        to [stroke diode, invert, -*] ++(-1.5,-1.5) coordinate(south)
        to [stroke diode, invert] (west);

    \draw(west)
        to [C,] ++(0,-4)
        node[rground](u1){};

    \draw(east)
        to [short] (east|-u1)
        node[rground](u2){};

    \draw(north)
        to [short] ++(0,1)
        to [short] ++(3,0) coordinate(o3)
        to [short] ++(3,0) coordinate(o4);

    \draw(south)
        to [short] ++(0,-1)
        to [short] ++(3,0) coordinate(u3)
        to [short] ++(3,0) coordinate(u4);

    \draw(o3)
        to [european resistor, *-] ++(0,-1.6666)
        to [variable european resistor, tunable end arrow={Bar}, invert, name={R2}] ++(0,-1.6666)
        to [european resistor, -*] (u3);

    \draw(R2.wiper)
        to [short, -o] (R2.wiper|-u2)
        node[anchor=west](){$f_\mathrm{OSZ}$};
    \ctikzset{
        quadpoles/transformer core/inner=1,
        quadpoles/transformer core/width=0.6,
        quadpoles/transformer core/height=2,
        transformer L1/.style={inductors/coils=6},
        transformer L2/.style={inductors/coils=3, inductors/width=0.5}    
    }
    \draw(u4|-east)
        node[transformer core, anchor=west](T){};

    \ctikztunablearrow[-{Bar}]{2}{1.8}{155}{T-L1}

    \draw(o4)
        to [short, -*] (T.A1)
        to [short] ++(-1.5,0) coordinate(h1)
        to [C,] (h1|-T.A2)
        to [short, -*] (T.A2);

    \draw(T.A2)
        to [short, -*] (u4)
        to [vC, tunable end arrow={Bar}, invert] (u4|-u2)
        node[rground](){};

    \draw(T.B2)
        to [short] (T.B2|-u2)
        node[rground](){};

    \draw(T.B1)
        to [short, -o] ++(1,0) coordinate(m17);

    \draw(m17)
        node[rotate=90, anchor=east](){Pufferstufe};

    \draw(o4)
        to [C,*-] ++(0,1.5)
        node[rground, rotate=180](){};
\end{circuitikz}