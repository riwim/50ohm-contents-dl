% Author: Prof. Dr. Matthias Jung, DL9MJ
% Year: 2024
\begin{circuitikz}
    \ctikzset{
        resistors/scale=\getDarcImageFactor,
        capacitors/scale=\getDarcImageFactor,
        inductors/scale=\getDarcImageFactor,
        blocks/scale=\getDarcImageFactor,
        RF/scale=\getDarcImageFactor,
        switches/scale=\getDarcImageFactor,
        misc/scale=\getDarcImageFactor
    }
    \draw(0,0) node[antenna]{}
        to [bandpass, box, >, l_=HF, name={bp1}] ++ (2,0)
        node[inputarrow](){}
        node[mixer, box, anchor=west] (mix1) {};
    \draw(mix1.east)
        to [bandpass, box, >, l_=1.ZF, name={bp2}] ++ (2,0)
        node[inputarrow](){}
        node[mixer, box, anchor=west] (mix2) {};
    \draw(mix2.east)
        to [bandpass, box, >, l_=2.ZF, name={bp3}] ++ (2,0)
        node[inputarrow](){}
        node[mixer, box, anchor=west](mix3){};
    \draw(mix3.east)
        to [bandpass, box, >, l_=3.ZF, name={bp4}] ++ (2,0)
        node[inputarrow](){};

    \draw(mix1.south) node[inputarrow, rotate=90] {} 
        to [short] ++(0,-1)
        node [vallpassshape, anchor=north](x){};

    \draw(mix2.south) node[inputarrow, rotate=90] {} 
        to [short] ++(0,-1)
        node [qgeneratorshape, t={G}, anchor=north](y){};

    \draw(mix3.south) node[inputarrow, rotate=90] {} 
        to [short] ++(0,-1)
        node [qgeneratorshape, t={G}, anchor=north](z){};

    \draw(x.east)  node[anchor=west](){VFO};
    \draw(y.south) node[anchor=north](){\qty{41}{\mega\hertz}};
    \draw(z.south) node[anchor=north](){\qty{9,455}{\mega\hertz}};

    \draw(bp1.north) node[anchor=west,rotate=90](){\qty{3}-\qty{30}{\mega\hertz}};
    \draw(bp2.north) node[anchor=west,rotate=90](){\qty{50}{\mega\hertz}};
    \draw(bp3.north) node[anchor=west,rotate=90](){\qty{9}{\mega\hertz}};
    \draw(bp4.north) node[anchor=west,rotate=90](){\qty{455}{\kilo\hertz}};
\end{circuitikz}