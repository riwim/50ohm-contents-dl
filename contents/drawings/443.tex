% Author: Dr. Matthias Jung, DL9MJ
% Year: 2022
\begin{circuitikz}
    \draw(0,4) coordinate(o0)
        to [short, o-*] ++(1,0) coordinate(n1)
        to [rmeter, t=2, -*] ++(3,0) coordinate(n2)
        to [rmeter, t=4, -o] ++(3,0) coordinate(o1);
    \draw(0,0) coordinate(u0)
        to [short, o-*] ++(1,0) coordinate(n3)
        to [short, -*] ++(3,0) coordinate(n4)
        to [short, -o] ++(3,0) coordinate(u1);
    \draw(n1) to [rmeter, t=1, name={h1}] (n3); 
    \draw(n2)
        to [rmeter, t=3] ++(0,-2)
        node [ampshape, box, anchor=north](amp){};
    \draw(amp.south) to [short] (n4); 
    \draw(amp.east) node [anchor=west] {PA};
    \draw(o0|-h1.center) node [rotate=90] {zum Netzteil};
    \draw(o1|-h1.center) node [rotate=90] {zum TX};
\end{circuitikz}